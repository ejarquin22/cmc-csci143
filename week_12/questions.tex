\documentclass{exam}

\usepackage[margin=1in]{geometry}
\usepackage{amsmath}
\usepackage{amssymb}
\usepackage{amsthm}
\usepackage{mathtools}
\usepackage{bm}

\usepackage{color}
\usepackage{colortbl}
\definecolor{deepblue}{rgb}{0,0,0.5}
\definecolor{deepred}{rgb}{0.6,0,0}
\definecolor{deepgreen}{rgb}{0,0.5,0}
\definecolor{gray}{rgb}{0.7,0.7,0.7}

\usepackage{hyperref}
\hypersetup{
  colorlinks   = true, %Colours links instead of ugly boxes
  urlcolor     = black, %Colour for external hyperlinks
  linkcolor    = blue, %Colour of internal links
  citecolor    = blue  %Colour of citations
}

\usepackage{listings}
\lstset {
	basicstyle=\ttfamily,
    ,language=SQL
    ,showstringspaces=false
    ,keepspaces=true
}

%%%%%%%%%%%%%%%%%%%%%%%%%%%%%%%%%%%%%%%%%%%%%%%%%%%%%%%%%%%%%%%%%%%%%%%%%%%%%%%%

%\printanswers

% Create a True False question format
\newcommand*{\TrueFalse}[1]{%
\ifprintanswers
    \ifthenelse{\equal{#1}{T}}{%
        \textbf{TRUE}\hspace*{14pt}False
    }{
        True\hspace*{14pt}\textbf{FALSE}
    }
\else
    {True}\hspace*{20pt}False
\fi
} 
%% The following code is based on an answer by Gonzalo Medina
%% https://tex.stackexchange.com/a/13106/39194
\newlength\TFlengthA
\newlength\TFlengthB
\settowidth\TFlengthA{\hspace*{1.16in}}
\newcommand\TFQuestion[2]{%
    \setlength\TFlengthB{\linewidth}
    \addtolength\TFlengthB{-\TFlengthA}
    \parbox[t]{\TFlengthA}{\TrueFalse{#1}}\parbox[t]{\TFlengthB}{#2}
}


\theoremstyle{definition}
\newtheorem{problem}{Problem}
\newtheorem{defn}{Definition}
\newcommand{\R}{\mathbb R}
\DeclareMathOperator{\E}{\mathbb E}
\DeclareMathOperator{\nnz}{nnz}
\DeclareMathOperator{\determinant}{det}
\DeclareMathOperator{\Var}{Var}
\DeclareMathOperator{\rank}{rank}
\DeclareMathOperator*{\argmin}{arg\,min}
\DeclareMathOperator*{\argmax}{arg\,max}

\newcommand{\I}{\mathbf I}
\newcommand{\Q}{\mathbf Q}
\newcommand{\p}{\mathbf P}
\newcommand{\pb}{\bar {\p}}
\newcommand{\pbb}{\bar {\pb}}
\newcommand{\pr}{\bm \pi}

\newcommand{\trans}[1]{{#1}^{T}}
\newcommand{\loss}{\ell}
\newcommand{\w}{\mathbf w}
\newcommand{\x}{\mathbf x}
\newcommand{\y}{\mathbf y}
\newcommand{\lone}[1]{{\lVert {#1} \rVert}_1}
\newcommand{\ltwo}[1]{{\lVert {#1} \rVert}_2}
\newcommand{\lp}[1]{{\lVert {#1} \rVert}_p}
\newcommand{\linf}[1]{{\lVert {#1} \rVert}_\infty}
\newcommand{\lF}[1]{{\lVert {#1} \rVert}_F}

% Shalev-Swartz and Ben-David

\newcommand{\h}{\mathcal H}
\newcommand{\D}{\mathcal D}
\DeclareMathOperator*{\erm}{ERM}

%%%%%%%%%%%%%%%%%%%%%%%%%%%%%%%%%%%%%%%%%%%%%%%%%%%%%%%%%%%%%%%%%%%%%%%%%%%%%%%%

\begin{document}


\begin{center}
\Huge
Questions on Indexes and Table Storage
\end{center}


\section{True/False Questions}

For each question below, circle either True or False.
On your final exam,
each correct answer will result in +1 point,
each incorrect answer will result in -1 point,
and each blank answer in 0 points.
For this homework assignment, you can uncomment the following line in the tex file to view the answers:
\begin{verbatim}
printanswers
\end{verbatim}
and so these questions do not need to be submitted.
You should still try to complete them, however, to check your understanding.
Approximately 4/5 of these questions are answered in class,
and the remaining 1/5 you'll have to refer to the postgres documentation / supplementary material for the answers.

\begin{questions}

%%%%%%%%%%%%%%%%%%%%%%%%%%%%%%%%%%%%%%%%%%%%%%%%%%%%%%%%%%%%%%%%%%%%%%%%%%%%%%%%
\fullwidth{\textbf{Hash Index}}

\question\TFQuestion{F}{A hash index can return columns in sorted order.}
\question\TFQuestion{T}{A has index on a text column whose average length is 100s of characters will use less disk space than an equivalent btree index.}
\question\TFQuestion{F}{A table can be CLUSTERed on a hash index.}
\question\TFQuestion{T}{A hash index can be used for index scans.}
\question\TFQuestion{T}{A hash index can be used to speed up a nested loop join.}
\question\TFQuestion{F}{A hash index can be used to speed up a merge join.}
\question\TFQuestion{F}{A hash index can be used to speed up a hash join.}
\question\TFQuestion{F}{The main advantage of a hash index over a btree index is that a hash index can result in fewer TABLE page accesses.}
\question\TFQuestion{T}{The main advantage of a hash index over a btree index is that a hash index can result in fewer INDEX page accesses.}

%%%%%%%%%%%%%%%%%%%%%%%%%%%%%%%%%%%%%%%%%%%%%%%%%%%%%%%%%%%%%%%%%%%%%%%%%%%%%%%%

\fullwidth{\textbf{GIN Index}}

\question\TFQuestion{F}{The GIN index supports index only scans.}
\question\TFQuestion{F}{The GIN index supports index scans.}
\question\TFQuestion{T}{The GIN index supports bitmap scans.}
\question\TFQuestion{F}{A table can be CLUSTERed on a GIN index.}
\question\TFQuestion{F}{You have a SELECT query that returns hundreds of thousands of rows.  Postgres is using a GIN index to speed up the query, but it is still taking a long time.  The query could be sped up dramatically by adding a LIMIT clause to reduce the number of rows returned.}
\question\TFQuestion{T}{A GIN index can created on multiple columns.}
\question\TFQuestion{F}{A GIN index can be used to speed up merge joins if the join condition is constructed appropriately.}
\question\TFQuestion{F}{A GIN index can be used to speed up hash joins if the join condition is constructed appropriately.}
\question\TFQuestion{T}{A GIN index can be used to speed up nested loop joins if the join condition is constructed appropriately.}
\question\TFQuestion{F}{A GIN index created on a \lstinline{tsvector} stores information about the position of lexemes within the document.}
\question\TFQuestion{F}{If Postgres crashes while a DELETE/INSERT/UPDATE statement is modifying a GIN index, the index becomes corrupted and must be regenerated from scratch.}

%%%%%%%%%%%%%%%%%%%%%%%%%%%%%%%%%%%%%%%%%%%%%%%%%%%%%%%%%%%%%%%%%%%%%%%%%%%%%%%%

\fullwidth{\textbf{RUM Index}}

\question\TFQuestion{F}{The RUM index supports index only scans.}
\question\TFQuestion{T}{The RUM index supports index scans.}
\question\TFQuestion{T}{The RUM index supports bitmap scans.}
\question\TFQuestion{T}{You have a SELECT query that returns hundreds of thousands of rows.  Postgres is using a RUM index to speed up the query, but it is still taking a long time.  The query could be sped up dramatically by adding a LIMIT clause to reduce the number of rows returned.}
\question\TFQuestion{F}{A table can be CLUSTERed on a RUM index.}
\question\TFQuestion{F}{A RUM index can be used to speed up merge joins if the join condition is constructed appropriately.}
\question\TFQuestion{F}{A RUM index can be used to speed up hash joins if the join condition is constructed appropriately.}
\question\TFQuestion{T}{A RUM index can be used to speed up nested loop joins if the join condition is constructed appropriately.}
\question\TFQuestion{T}{The RUM index uses more disk space than the GIN index.}
\question\TFQuestion{T}{A RUM index created on a \lstinline{tsvector} stores information about the position of lexemes within the document.}
\question\TFQuestion{T}{RUM indexes do not support the \lstinline{fastupdate} index creation parameter, and therefore inserting on a RUM index is slower than on a GIN index.}
\question\TFQuestion{F}{If Postgres crashes while a DELETE/INSERT/UPDATE statement is modifying a RUM index, the index becomes corrupted and must be regenerated from scratch.}

%%%%%%%%%%%%%%%%%%%%%%%%%%%%%%%%%%%%%%%%%%%%%%%%%%%%%%%%%%%%%%%%%%%%%%%%%%%%%%%%

\fullwidth{\textbf{Unicode}}

\question\TFQuestion{T}{Postgresql's implementation of UTF-8 is complies with the Unicode standard.}
\question\TFQuestion{T}{Emojis can be stored in a TEXT column if the database is using UTF-8 encodings.}
\question\TFQuestion{T}{Given any string in NFC form, normalizing to NFD and back to NFC is guaranteed to be an idempotent operation (i.e. you will get the same string back.)}
\question\TFQuestion{F}{Given any string in NFKD form, normalizing to NFD and back to NFKD is guaranteed to be an idempotent operation (i.e. you will get the same string back.)}
\question\TFQuestion{T}{Given the string ``C\'esar Ch\'avez'', an NFC-normalized UTF-8 encoding will require fewer bytes than a NFD-normalized UTF-8 encoding.}
\question\TFQuestion{T}{Postgres can compress TEXT columns no matter what language is contained.}
\question\TFQuestion{F}{Postgres will automatically normalize all text into NFC form.}
\question\TFQuestion{T}{All characters from the Klingon writing system can be represented in Unicode.}
\question\TFQuestion{F}{NFD is a system for encoding Unicode code points as bytes.}
\question\TFQuestion{F}{ANSI is a system for encoding Unicode code points as bytes.}
\question\TFQuestion{T}{ASCII is a system for encoding Unicode code points as bytes.}
\question\TFQuestion{T}{UTF-8 is a system for encoding Unicode code points as bytes.}
\question\TFQuestion{T}{UTF-16 is a system for encoding Unicode code points as bytes.}

%%%%%%%%%%%%%%%%%%%%%%%%%%%%%%%%%%%%%%%%%%%%%%%%%%%%%%%%%%%%%%%%%%%%%%%%%%%%%%%%

\fullwidth{\textbf{Full Text Search}}

\question\TFQuestion{T}{Postgresql's built-in \lstinline{to_tsvector} function has support for the Arabic language.}
\question\TFQuestion{F}{Postgresql's built-in \lstinline{to_tsvector} function has support for the Korean language.}
\question\TFQuestion{T}{Postgresql's built-in \lstinline{to_tsvector} function has support for the Spanish language.}
\question\TFQuestion{F}{Postgresql's built-in \lstinline{pg_trgm} allows searching Chinese text.}
\question\TFQuestion{T}{A GIN index built using bigrams generated from the \lstinline{pg_bigm} extension will be crash-safe (i.e. the index will be valid if postgres crashes during an INSERT/UPDATE/DELETE operation).}
\question\TFQuestion{F}{A pgroonga index will be crash safe (i.e. the index will be valid if postgres crashes during an INSERT/UPDATE/DELETE operation).}
\question\TFQuestion{T}{A GIN index created with the \lstinline{gin_trgm_ops} operator class supports faster searches using the LIKE operator.}
\question\TFQuestion{F}{A GIN index created with the \lstinline{gin_trgm_ops} operator class supports faster searches using the \lstinline{@@} operator.}
\question\TFQuestion{F}{A GIN index created with the \lstinline{tsvector_ops} operator class supports faster searches using the LIKE operator.}
\question\TFQuestion{T}{A GIN index created with the \lstinline{tsvector_ops} operator class supports faster searches using the \lstinline{@@} operator.}


\end{questions}
\end{document}



